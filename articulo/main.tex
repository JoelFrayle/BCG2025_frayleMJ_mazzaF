\documentclass[journal]{IEEEtran}
\usepackage{enumerate}
\usepackage{stfloats}
\usepackage{graphicx}
\usepackage{hyperref}
\usepackage{multirow}
\usepackage{listings}

\lstset{
    basicstyle=\ttfamily\small,
    numbers=left,       % Números de línea a la izquierda
    numberstyle=\color{gray},
    frame=single,       % Borde alrededor del código
    breaklines=true,    % Cortar líneas largas
    showstringspaces=false,
    language=bash       % Lenguaje (bash para scripts Unix)
}
%\usepackage[USenglish,british,american,australian,english]{babel}
\usepackage[spanish, activeacute]{babel} %Change language
\usepackage{float}%For static figures
\graphicspath{{Images/}}%Set folder to look for images
\makeatletter
\usepackage{url}%Print URLS nicely
\g@addto@macro{\UrlBreaks}{\UrlOrds}%Break urls into lines
\let\runtitle\@title
\makeatother
\usepackage[cmex10]{amsmath}
\usepackage{amssymb}
\usepackage{gensymb}
\usepackage{comment}
%\usepackage{breqn}
\usepackage[utf8]{inputenc}%Accept tildes
\begin{document}

%%%%%%%%%%%%%%%%%%%%%%%%%%%%%%%%%%%%%%%%%%%%%%%%%%%%%%%%%%%%%%%%%%%%%%%%%%%%%%%%


\title{Modelo de 2 serotipos de transmisión de Dengue: métodos de control y estrategias de vacunación}
\makeatletter
\let\Title\@title
\makeatother
\author{
\IEEEauthorblockN{
Joel Frayle Moreno, 
José Gómez Bonilla,
Juan Pablo Pérez,  
Santiago Rodríguez Rodríguez 
}\\
\IEEEauthorblockA{
Email: 
\IEEEauthorrefmark{1}j.frayle@uniandes.edu.co,
\IEEEauthorrefmark{2}jf.gomezb1@uniandes.edu.co,
\IEEEauthorrefmark{3}jp.perezu@uniandes.edu.co,
\IEEEauthorrefmark{4} s.rodriguezr2@uniandes.edu.co
}}
\markboth{IBIO4111 GRUPO 9 (2022)}
%\markboth{\Title}
{\Title}
\maketitle



%%%%%%%%%%%%%%%%%%%%%%%%%%%%%%%%%%%%%%%%%%%%%%%%%%%%%%%%%%%%%%%%%%%%%%%%%%%%%%%%
% INTRODUCCIÓN

\section{\textbf{Introducción y Marco Teórico}}  

\href{https://github.com/tuusuario/repositorio/blob/main/genetic_analysis.sh#L10}{Ver línea 10 del código}
%Contexto del problema y cifras relevantes
El virus del Dengue (DENV) es la infección arboviral de más rápida expansión en el mundo, causando un estimado de 390 millones de infecciones y 60 millones de casos sintomáticos cada año a nivel mundial. Se ha estimado además que aproximadamente 40\% de la población mundial vive en áreas tropicales y subtropicales en riesgo de transmisión de DENV, lo que corresponde a 2500 millones de personas en más de 100 países. Además, una gran proporción de las infecciones son asintomáticas o causan un síndrome febril leve, por lo que no es posible cuantificar la transmisión utilizando la incidencia de los casos clínicamente sospechosos, como suelen notificar los sistemas de vigilancia, ya que sólo representa una fracción de las infecciones \cite{article1}.
La tendencia a la expansión mundial y el aumento de la carga de la enfermedad, sobre todo en los países con recursos limitados, ha vuelto al dengue un problema de salud pública que necesita de alta prioridad \cite{Dengue_Col}.

La enfermedad se transmite por la picadura de la hembra infectada del mosquito \textit{Aedes aegypti} o \textit{Aedes albopictus}. Para transmitir la enfermedad el mosquito debe haber picado a una persona infectada durante el período de viremia, que ocurre después de un período de incubación de aproximadamente 7-10 días. Además, Se ha estimado que el periodo de incubación de la enfermedad en el cuerpo del mosquito es de 4 a 6 días \cite{article}. %Revisar 

El mosquito tiene una fase acuática (Huevo-larva-Pupa) en la cual no es infeccioso. Sumado a esto, la transmisión se puede dar de huésped a huésped por efectos de transfusión de sangre, trasplante de órganos y transmisión vertical \cite{article3}. 

El virus tiene 4 serotipos (DEN 1-4). Los individuos infectados con un serotipo mantienen una inmunidad protectora de por vida contra la infección por el virus homólogo, pero la inmunidad protectora contra la infección por serotipos heterólogos es temporal. Cuando la protección cruzada a corto plazo disminuye, los pacientes que experimentan una infección secundaria por otro serotipo tienen un mayor riesgo de enfermedad grave, a través del proceso de amplificación dependiente de anticuerpos (ADE). Además, debido al corto periodo de vida de los mosquitos, estos nunca se recuperan del virus. Por último, se ha reportado que las estaciones y la temperatura influyen sobre la dinámica de infección \cite{aguiar_anam_blyuss_estadilla_guerrero_knopoff_kooi_srivastav_steindorf_stollenwerk_2022}.

Diversos estudios han identificado varios métodos de control potencialmente efectivos contra la propagación del dengue. Dentro de estos, destaca la disposición de una vacuna tetravalente contra este virus \cite{pang_mak_gubler_2017, quintero_garcia-betancourt_cortes_garcia_alcala_gonzalez-uribe_brochero_carrasquilla_2015}. No obstante, las pocas vacunas que han logrado este nivel de inmunización (tal como la TAK-003) no se encuentran ampliamente distribuidas en Colombia \cite{claypool_brandeau_goldhaber-fiebert_2021}.  Por este motivo, otros métodos de control -tales como insecticidas, barreras mecánicas (mallas) y campañas de concientización- han tomado importancia en el control del dengue; sin embargo, es necesario investigar con mayor detalle su eficacia en diversas zonas del país. \cite{pang_mak_gubler_2017, quintero_garcia-betancourt_cortes_garcia_alcala_gonzalez-uribe_brochero_carrasquilla_2015, santos_parra-henao_silva_augusto_2014}.

%Pregunta de investigación:
Este proyecto está centrado en la población de Piedecuesta, Santander, pues se han encontrado estudios epidemiológicos enfocados en esta zona que facilitan la escogencia de valores para los parámetros de las ecuaciones. Asimismo, según el ministerio de salud y SIVIGILIA esta región es de riesgo moderado y hace parte de los municipios con transmisión endémica persistente que acumulan entre el 50\% y 70\% de los casos en Colombia \cite{article1}. Según el Instituto Nacional de Salud, los serotipos presentes en esta región son DENV1 y DENV2, con una mayor incidencia de DENV2 (10:7) \cite{insti, perez} . %Con una mayor indicidencia de DENV2 que DENV1      

Teniendo en cuenta lo descrito anteriormente se plantea la siguiente pregunta de investigación:
¿Qué método o combinación de métodos de control disminuye la propagación de  DENV1 y DENV2 en mayor medida en la zona urbana de Pie de Cuesta Santander?

% Métodos de control:
El objetivo de este proyecto es responder a esta pregunta para brindar recomendaciones a la alcaldía de Piedecuesta que puedan ser aplicadas para reducir los casos de dengue en la región. Para este fin, el desarrollo de un modelo matemático es una buena alternativa, pues permite visualizar simultáneamente la dinámica poblacional de humanos, mosquitos y larvas, con y sin controles, y estudiar su estabilidad y equilibrios. Entonces, se propone un modelo SEIR para la dinámica de humanos y uno SEI para mosquitos para evaluar el comportamiento de los sistemas acoplados al incluir métodos de control mecánicos, químicos y de vacunación. Adicionalmente, se realizaran simulaciones en donde los parámetros del modelo cambien respecto a la favorabilidad que aportan las condiciones climáticas al crecimiento de la población de mosquitos. Este diseño fue inspirado en los modelos propuestos en \cite{carvalho2019mathematical}, \cite{Maidana_Yang_2008} y \cite{janreung_chinviriyasit_chinviriyasit_2020}.

% Justificación del plantamiento
Para dar respuesta a la pregunta planteada se comparó la población de humanos infectados del modelo nulo respecto a los controles para definir qué método o combinación de métodos disminuye más rápido y en mayor medida la cantidad de humanos infectados tras un año, ya que en este tiempo logra observarse en qué momento la enfermedad tiende a extinguirse. Para ello, el método de medición utilizado es la dinámica de humanos infectados, su pico máximo y su valor al final del período Diciembre-Septiembre. Adicionalmente, se estudia la dinámica de mosquitos infectados como complemento, pero no como método de decisión, pues incluso cuando los mosquitos infectados se eliminan del sistema, los susceptibles todavía están presentes y los humanos infectados provocan la reaparición de mosquitos infectados \cite{carvalho2019mathematical}. %Revisar lo del año.


%%%%%%%%%%%%%%%%%%%%%%%%%%%%%%%%%%%%%%%%%%%%%%%%%%%%%%%%%%%%%%%%%%%%%%%%%%%%%%%%
\section{Datos}

Se cuenta con datos epidemiológicos de Piedecuesta recuperados del Instituto Nacional de Salud y SIVIGILA \cite{insti,article1}. principalmente para definir los valores iniciales poblacionales y la incidencia de cada serotipo. Los valores de los parámetros se escogieron con base a vigilancia bibliográfica de diversos artículos y se presentan en las figuras \ref{Fi:valores modelo nulo1},\ref{Fi:valores modelo nulo2},\ref{Fi:valores modelo nulo3},\ref{Fi:valores modelo nulo4}.


%%%%%%%%%%%%%%%%%%%%%%%%%%%%%%%%%%%%%%%%%%%%%%%%%%%%%%%%%%%%%%%%%%%%%%%%%%%%%%%%
\section{Cálculos y Modelo Matemático}

\subsection{Modelo Nulo}
En las figuras \ref{fig:modelo1}, \ref{fig:modelo2}  se presenta el diagrama de bloques del sistema nulo. Las 3 primeras etapas en la vida del mosquito corresponden a estados acuáticos en los que no adquiere la enfermedad (Huevo, larva, pupa). Por tanto, para efectos del modelo, estos 3 estados se agrupan en una única fase acuática (A) que de aquí en adelante se llamará larvas. Se asume que todos los mosquitos en estado larva se convierten en adultos susceptibles a una tasa $\gamma$ y el crecimiento de los mosquitos estará limitado por la capacidad de carga del crecimiento logístico de las larvas. El crecimiento logístico estará determinado por la tasa de oviposición $r$, y la cantidad de larvas que nacen estará limitada por la capacidad de carga $k$, la cual refleja el recurso hídrico disponible. Un mosquito susceptible ($M_S$) pasará a ser un mosquito expuesto ($M_E$) cuando pique a un humano infectado a una tasa de contacto efectivo entre humanos infectados y mosquitos susceptibles dada por $\beta_M$. Los subíndices 1 y 2 de cada variable representan los serotipos DENV1 y DENV2 respectivamente. Finalmente un mosquito expuesto pasará a ser un mosquito infeccioso después del periodo de incubación $\sigma_m$. Los mosquitos adultos morirán a una tasa $\mu_1$ y las larvas a una tasa $\mu_2$.

En la figura \ref{fig:modelo2}´se evidencian las dinámicas que tendrá la población de humanos. Un humano susceptible ($H_S$) pasará a ser un humano expuesto ($H_E$) al virus cuando sea picado por un mosquito infeccioso, dado por una tasa de contacto efectivo entre mosquitos infecciosos y humanos susceptibles representada por  $\beta_H$. Pasado el periodo de incubación  $\sigma_H$ los humanos expuestos pasaran a ser humanos infectados ($H_I$). Finalmente, los humanos infectados se recuperarán a una tasa $\alpha$ y obtendrán inmunidad permanente al primer serotipo con el que se infecten e inmunidad cruzada hacia el otro serotipo, modelada por un porcentaje de reducción $0<\lambda<1$ de la tasa de contacto $\beta_H$. Tras este período los humanos pueden estar expuestos a una infección secundaria ($H_E12$, $H_E21$ : $H_I12$, $H_I21$) tras la cuál se recuperarán con inmunidad permanente a ambos serotipos. La tasa de mortalidad natural para todos los humanos sera $\mu_H$ y será igual a su tasa de natalidad para obtener una población constante.

\begin{figure}[H]
\centering
\includegraphics[width=0.40\textwidth]{articulo/modelo1.png}
\caption{Modelo SEI mosquitos}
\label{fig:modelo1}
\centering
\end{figure}


\begin{figure}[H]
\centering
\includegraphics[width=0.40\textwidth]{articulo/modelo2.png}
\caption{Modelo SEIR Humanos}
\label{fig:modelo2}
\centering
\end{figure}

Adicionalmente, se resalta que los parámetros anteriores van a estar definidos dentro del sistema nulo siguiendo las suposiciones que se muestran a continuación:
\begin{itemize}
    \item La infección sucede de vector a huésped y de huésped a vector.  
    \item Los mosquitos infecciosos no vuelven a ser susceptibles debido a su corto periodo de vida. 
    \item Se tienen en cuenta 2 serotipos. Hay inmunidad cruzada, pero no ADE (mejora dependiente de anticuerpos).
    \item Los períodos de incubación de DENV1 y DENV2 en mosquitos son 6 días y 4 días ($\sigma_M$) y 10 y 7 días en humanos ($\sigma_H$).
    \item Dinámicas vitales: Tasa de crecimiento y muerte humanos:
    \begin{itemize}
      \item{Solo hay una tasa de muerte para humanos que tiene en cuenta la muerte natural.}
      \item{No se tiene en cuenta mortalidad asociada al virus.}
    \end{itemize}
    \item Se asume que todos los humanos nacen susceptibles a la enfermedad.
    \item Población de humanos constante $N=S+E+I+R$. Es decir, la tasa de muerte (muerte natural y emigración) es igual a la tasa de crecimiento (natalidad, inmigración).
    \item Las dinámicas de transmisión de la enfermedad se modelan teniendo en cuenta la ley de masas.
\end{itemize}

Las ecuaciones del modelo nulo se encuentran en anexos.

Adicionalmente, se dividió el año en 3 periodos considerando la favorabilidad de la lluvia. El periodo favorable corresponde a los meses de Diciembre-Marzo, Intermedio de Abril-Mayo y Desfavorable de Junio-Septiembre. Durante cada período cambian los parámetros de tasa de conversión de larvas a mosquitos adultos $\gamma$, la tasa de mortalidad de los mosquitos adultos $\mu_1$ y la capacidad de carga $K_{0}$. Las condiciones iniciales y valores de los parámetros del modelo nulo se encuentran en los anexos \ref{Fi:valores modelo nulo1}, \ref{Fi:valores modelo nulo2}, \ref{Fi:valores modelo nulo3}, \ref{Fi:valores modelo nulo4} 
\subsection{Modelo con controles}

\subsubsection{Controles mecánicos}

Los controles mecánicos corresponden a las estrategias implementadas para disminuir las zonas de criaderos de larvas, al implementar este control en el modelo la capacidad de carga es modificada por $K = K_i * K_0$. Donde $K_0$ es la capacidad de carga en ausencia de controles que varía de acuerdo con el clima y $K_i$ es un número adimensional entre 0 y 1 que indica la proporción de $K_0$ que disminuye con relación a los controles mecánicos.

\subsubsection{Controles químicos}

Para los controles químicos se implementaron dos controles al tiempo: insecticidas, que suman un valor $\mu_i$ al  parámetro la tasa de mortalidad de mosquitos adultos $\mu_1$ y larvicidas, que suman un valor $\mu_l$ al  parámetro de la tasa de mortalidad de larvas $\mu_2$. Estas tasas se modelan como funciones que decaen exponencialmente con el tiempo a partir de la liberación de sustancias químicas. Se asume que los químicos se liberan durante la primera semana de cada mes y que persisten por 15 días en el ambiente. En el primer día la liberación es máxima y como máximo el 10\% de la cantidad total de sustancias químicas liberadas permanece en el medio ambiente. Estas funciones también varían con el periodo de lluvias. 

\subsubsection{Vacunación}

Al implementar este control se brinda inmunidad permanente a los humanos susceptibles. Al adaptar esta situación al modelo propuesto, los humanos susceptibles pasan a una tasa $v$ a la población de humanos recuperados. En este caso se hace la suposición de que la vacuna es 100\% efectiva.

En el anexo \ref{Fi:diag_controles1}, \ref{Fi:diag_controles2} se observa el diagrama de cajas obtenido al implementar los controles. En azul se subraya el efecto de los controles mecánicos, en verde los químicos y en amarillo la vacunación. Adicionalmente, en la sección de anexos en la tabla \ref{Fi:val_controles} se encuentras los valores de los parámetros y variaciones asociadas a cada uno de los controles.

%%%%%%%%%%%%%%%%%%%%%%%%%%%%%%%%%%%%%%%%%%%%%%%%%%%%%%%%%%%%%%%%%%%%%%%%%%%%%%%%
\section{Análisis y discusión de resultados}

En una población cerrada en la que la tasa de nacimiento es igual a la de muerte, existe una proliferación del dengue entre los primeros 30 a 40 días de la transmisión de la enfermedad. Esto se evidencia en muchos estudios dentro del que se destaca el realizado por Side y colaboradores \cite{medan} en el que evaluaron la transmisión de la fiebre del dengue en Medan, una provincia de Indonesia con condiciones climáticas semejantes a Piedecuesta, Santander. Obtuvieron que la dinámica poblacional consiste en una disminución en el número de susceptibles la cual es proporcional al aumento de los casos de expuestos. Hacia el primer tercio del periodo de simulación se observa que el número de humanos que han sido picados por mosquitos es cercano al tercio de la población total, seguidamente se da un pico de infectados el cual acontece cuatro días después del pico de expuestos que se encuentra asociado al periodo de incubación del virus en el cuerpo; luego se da una disminución en los infectados que llega a su mínimo al rededor de los 50 días desde el inicio de la epidemia. Esto es acorde con los resultados del modelo obtenido que no tiene en cuenta algún tipo de control en la transmisión de la enfermedad. Se observa una disminución proporcional al número de recuperados durante el periodo del año correspondiente al periodo favorable, luego se estabiliza para los otros dos periodos del año. Como en el periodo favorable existe una alta proliferación de mosquitos, el número de vectores es suficiente para infectar a la mayoría de la población en los primeros cuatro meses de epidemia, se observa que existe un pico de expuestos hacia el día 51 y disminuye hasta cero al inicio del periodo intermedio. Respecto a los infectados, se observa un pico crítico cercano al día 60 que corresponde a 40710 personas, esto concuerda con los 10 días que fueron destinados como periodo de incubación del virus en el cuerpo una vez son picados por los mosquitos \ref{fig:resultado2} y anexo: \ref{fig:resultado1}. Por este alto número de infectados, es necesario implementar controles para la transmisión de la enfermedad con el fin de obtener la mejor relación tiempo-pico de la enfermedad.


\begin{figure}[H]
\centering
\includegraphics[width=0.40\textwidth]{articulo/Simulación modelo nulo 2.png}
\caption{Modelo SEIR nulo: transmisión de dengue en humanos para el periodo Diciembre-Septiembre de la suma de cada variable}
\label{fig:resultado2}
\centering
\end{figure}

\begin{figure}[H]
\centering
\includegraphics[width=0.40\textwidth]{articulo/Simulación modelo nulo 3.png}
\caption{Modelo SEI nulo: transmisión de dengue en mosquitos para el periodo Diciembre-Septiembre}
\label{fig:resultado3}
\centering
\end{figure}


Se realizaron también las simulaciones para cada control, y cada pareja de controles. Es decir, se realizó el modelo para el control mecánico, para el control mecánico + insecticida, control mecánico + larvicida, y así hasta completar cada pareja de controles. Esto se realizó con el fin de poder determinar que control o pareja de controles ofrecían el mejor método para controlar la expansión de la enfermedad.
\\Tras la experimentación se pudo determinar que el control que es más costo efectivo es el de procesos de fumigación con insecticidas constantes \ref{fig: tabla}. Para estudiar la variable de interés (Humanos infectados) se sumaron las variables de humanos infectados por cada serotipo. El efecto de la fumigación con insecticidas en los mosquitos resulta en una disminución de la población susceptible exponencial. Esto se puede ver en la Figura \ref{1J}, en el cual la población de mosquitos decrece muy rápido y no permite que haya contagio entre mosquitos. En la Figura \ref{2J} se puede observar que, por lo tanto, la población de humanos no tiene variaciones visibles en el número de susceptibles y recuperados, esto indica que si se realiza estas fumigaciones a lo largo de un periodo extenso como el modelado se podría incluso erradicar la población y llegar a un punto en el que los contagios a los humanos sea cercano a 0.


\begin{figure}[H]
    \centering
        \includegraphics[width=0.40\textwidth]{articulo/Tabla.png}
        \caption{Picos de humanos infectados por control.}
        \label{tabla}
    \centering
\end{figure}


\begin{figure}[H]
    \centering
        \includegraphics[width=0.40\textwidth]{articulo/Dengue mosquitos incecticida.png}
        \caption{Modelo SEI transmisión dengue entre mosquitos con uso de Insecticida frecuente.}
        \label{1J}
    \centering
\end{figure}

\begin{figure}[H]
    \centering
        \includegraphics[width=0.40\textwidth]{articulo/Dengue Humanos Insecticida.png}
        \caption{Modelo SEIR transmisión dengue entre Humanos con uso de Insecticida frecuente.}
        \label{2J}
    \centering
\end{figure}
Este modelo matemático fue comparado con el resto de las modelos para la combinación de métodos de prevención de la expansión del dengue, y tiene comportamientos similares a los de algunas combinaciones, sin embargo, dado que este es un método único significa menores costos que una combinación de métodos. Por lo cual, se puede afirmar que este sería el modelo recomendado para la alcaldía, ya que tiene una mejor costo efectividad que el resto de métodos y sus combinaciones.
%%%%%%%%%%%%%%%%%%%%%%%%%%%%%%%%%%%%%%%%%%%%%%%%%%%%%%%%%%%%%%%%%%%%%%%%%%%%%%%%
\section{Conclusión}

En cuanto al trabajo futuro se debe realizar un esfuerzo por reducir el sistema, de tal suerte que los equilibrios puedan converger y se pueda estudiar la estabilidad y calcular el número reproductivo básico. Además, se podría generalizar el modelo para n serotipos y sus interacciones. Estas interacciones han sido modeladas con base a la similaridad genética de los serotipos para definir su reactividad, inmunidad cruzada y mejora dependiente de anticuerpos \cite{genetica}. Otro factor a considerar a futuro es la falta de implementación de un plan de vacunación contra el Dengue en Colombia. La única vacuna autorizada para su distribución por la OMS es la Dengvaxia, la cual solo puede ser aplicada a personas mayores de 9 años que hayan sido contagiadas por el virus previamente. \cite{mattar2019historia} Por lo tanto, una investigación a futuro debería centrarse en identificar combinaciones de mecanismos de control distintos a la vacunación preventiva que permitan obtener resultados similares a los obtenidos con las campañas de vacunación en cuanto a la reducción de casos de Dengue en humanos. 
Sin embargo, gracias a este modelo se puede tener una aproximación a la realidad del comportamiento de la enfermedad sin ningún control y el comportamiento de las poblaciones sin se ejerce cada uno de los controles. Se puede concluir gracias a los modelos realizados que el método para recomendar y que mejor previene la aparición de la enfermedad con el menor costo de inversión es el uso de insecticida frecuente. Por tanto, para dar respuesta a la pregunta de investigación, el método para disminuir la propagación que lo logra en mayor medida para la población de la zona urbana de Pie de Cuesta Santander es el control químico por insecticida.
%%%%%%%%%%%%%%%%%%%%%%%%%%%%%%%%%%%%%%%%%%%%%%%%%%%%%%%%%%%%%%%%%%%%%%%%%%%%%%%%
% REFERENCIAS

\bibliographystyle{ieeetr}
\bibliography{refs}

\onecolumn

%%%%%%%%%%%%%%%%%%%%%%%%%%%%%%%%%%%%%%%%%%%%%%%%%%%%%%%%%%%%%%%%%%%%%%%%%%%%%%%%
\section{Anexos}

\subsection{Ecuaciones modelo nulo:}

Vectores:
\begin{equation}
    \label{eq1}
    M=M_S+M_{E1}+M_{E2}+M_{I1}+MI_2
\end{equation}

\begin{equation}
    \label{eq1}
    \frac{dA}{dt}=r\left(1-\frac{A}{K_0}\right)M-\mu_2A-\gamma A
\end{equation}

\begin{equation}
    \label{eq1}
    \frac{dM_S}{dt}=\gamma A-\mu_1M_S-\beta_{M1}M_S\frac{{(H}_{I1}+H_{I21})}{H\ }-\beta_{M2}M_S\frac{{(H}_{I2}+H_{I12})}{H\ }
\end{equation}

\begin{equation}
    \label{eq1}
    \frac{dM_{E1}}{dt}=\beta_{M1}M_S\frac{{(H}_{I1}+H_{I21})}{H\ }-\mu_1M_{E1}-\sigma_{M1}M_{E1}
\end{equation}

\begin{equation}
    \label{eq1}
    \frac{dM_{E2}}{dt}=\beta_{M2}M_S\frac{{(H}_{I2}+H_{I12})}{H\ }-\mu_2M_{E2}-\sigma_{M2}M_{E2}
\end{equation}

\begin{equation}
    \label{eq1}
    \frac{dM_{I1}}{dt}=\sigma_{M1}M_{E1}-\mu_1M_{I1}
\end{equation}

\begin{equation}
    \label{eq1}
    \frac{dM_{I2}}{dt}=\sigma_{M2}M_{E2}-\mu_2M_{I2}
\end{equation}

Humanos:

\begin{equation}
    \label{eq1}    H=H_s+H_{E1}+H_{E2}+H_{I1}+H_{I2}+H_{E12}+H_{I12}+H_{E21}+H_{I21}+H_{R1}+H_{R2}+H_R
\end{equation}

\begin{equation}
    \label{eq1}
    \frac{dH_S}{dt}=\mu_HH-\mu_HH_S-\beta_{H1}H_S\frac{M_{I1}}{M}-\beta_{H2}H_S\frac{M_{I2}}{M}
\end{equation}

\begin{equation}
    \label{eq1}
    \frac{dH_{E1}}{dt}=\beta_{H1}H_S\frac{M_{I1}}{M}-\sigma_{H1}H_{E1}-\mu_HH_{E1}
\end{equation}

\begin{equation}
    \label{eq1}
    \frac{dH_{E2}}{dt}=\beta_{H2}H_S\frac{M_{I2}}{M}-\sigma_{H2}H_{E2}-\mu_HH_{E2}
\end{equation}


\begin{equation}
    \label{eq1}
    \frac{dH_{I1}}{dt}=\sigma_{H1}H_{E1}-\mu_HH_{I1}-\alpha_1H_{I1}
\end{equation}

\begin{equation}
    \label{eq1}
    \frac{dH_{I2}}{dt}=\sigma_{H2}H_{E2}-\mu_HH_{I2}-\alpha_2H_{I2}
\end{equation}

\begin{equation}
    \label{eq1}
    \frac{dH_{R1}}{dt}={\alpha_1H_{I1}\ -\mu}_HH_{R1}-{\lambda\beta}_{H2}H_{R1}\frac{M_{I2}}{M}
\end{equation}


\begin{equation}
    \label{eq1}
    \frac{dH_{R2}}{dt}={\alpha_2H_{I2}\ -\mu}_HH_{R2}-{\lambda\beta}_{H1}H_{R2}\frac{M_{I1}}{M}
\end{equation}


\begin{equation}
    \label{eq1}
    \frac{dH_{E12}}{dt}={\lambda\beta}_{H2}H_{R1}\frac{M_{I2}}{M}-\mu_HH_{E12}-\sigma_{H2}H_{E12}
\end{equation}


\begin{equation}
    \label{eq1}
    \frac{dH_{E21}}{dt}={\lambda\beta}_{H1}H_{R2}\frac{M_{I1}}{M}-\mu_HH_{E21}-\sigma_{H1}H_{E21}
\end{equation}


\begin{equation}
    \label{eq1}
    \frac{dH_{I12}}{dt}=\sigma_{H2}H_{E12}-\mu_HH_{I12}-\alpha_2H_{I12}
\end{equation}


\begin{equation}
    \label{eq1}
    \frac{dH_{I21}}{dt}=\sigma_{H1}H_{E21}-\mu_HH_{I21}-\alpha_1H_{I21}
\end{equation}


\begin{equation}
    \label{eq1}
    \frac{dH_R}{dt}=\alpha_2H_{I12}+\alpha_1H_{I21}-\mu_HH_R
\end{equation}

%%%%%%%%%%%%%%%%%%%%%%%%%%%%%%%%%%%%%%%%%%%%%%%%%%%%%%%%%%%%%%%%%%%%%

\subsection{Ecuaciones modelo controles:}


Vectores:

\begin{equation}
    \label{eq1}
    \frac{dA}{dt}=r\left(1-\frac{A}{{K_i\ast K}_0}\right)M-{(\mu}_2+\mu_l)A-\gamma A
\end{equation}

\begin{equation}
    \label{eq1}
    \frac{dM_S}{dt}=\gamma A-{(\mu}_1+\mu_i)M_S-\beta_{M1}M_S\frac{{(H}_{I1}+H_{I21})}{H\ }-\beta_{M2}M_S\frac{{(H}_{I2}+H_{I12})}{H\ }
\end{equation}

\begin{equation}
    \label{eq1}
    \frac{dM_{E1}}{dt}=\beta_{M1}M_S\frac{{(H}_{I1}+H_{I21})}{H\ }-{(\mu}_1+\mu_i)M_{E1}-\sigma_{M1}M_{E1}
\end{equation}

\begin{equation}
    \label{eq1}
    \frac{dM_{E2}}{dt}=\beta_{M2}M_S\frac{{(H}_{I2}+H_{I12})}{H\ }-{(\mu}_2+\mu_i)M_{E2}-\sigma_{M2}M_{E2}
\end{equation}

\begin{equation}
    \label{eq1}
    \frac{dM_{I1}}{dt}=\sigma_{M1}M_{E1}-{(\mu}_1+\mu_i)M_{I1}
\end{equation}

\begin{equation}
    \label{eq1}
    \frac{dM_{I2}}{dt}=\sigma_{M2}M_{E2}-{(\mu}_1+\mu_i)M_{I2}
\end{equation}


Humanos:

\begin{equation}
    \label{eq1}
    \frac{dH_S}{dt}=\mu_HH-\mu_HH_S-\beta_{H1}H_S\frac{M_{I1}}{M}-\beta_{H2}H_S\frac{M_{I2}}{M}-vH_S
\end{equation}

\begin{equation}
    \label{eq1}
    \frac{dH_{E1}}{dt}=\beta_{H1}H_S\frac{M_{I1}}{M}-\sigma_{H1}H_{E1}-\mu_HH_{E1}
\end{equation}

\begin{equation}
    \label{eq1}
    \frac{dH_{E2}}{dt}=\beta_{H2}H_S\frac{M_{I2}}{M}-\sigma_{H2}H_{E2}-\mu_HH_{E2}
\end{equation}


\begin{equation}
    \label{eq1}
    \frac{dH_{I1}}{dt}=\sigma_{H1}H_{E1}-\mu_HH_{I1}-\alpha_1H_{I1}
\end{equation}

\begin{equation}
    \label{eq1}
    \frac{dH_{I2}}{dt}=\sigma_{H2}H_{E2}-\mu_HH_{I2}-\alpha_2H_{I2}
\end{equation}

\begin{equation}
    \label{eq1}
    \frac{dH_{R1}}{dt}={\alpha_1H_{I1}\ -\mu}_HH_{R1}-{\lambda\beta}_{H2}H_{R1}\frac{M_{I2}}{M}
\end{equation}


\begin{equation}
    \label{eq1}
    \frac{dH_{R2}}{dt}={\alpha_2H_{I2}\ -\mu}_HH_{R2}-{\lambda\beta}_{H1}H_{R2}\frac{M_{I1}}{M}
\end{equation}


\begin{equation}
    \label{eq1}
    \frac{dH_{E12}}{dt}={\lambda\beta}_{H2}H_{R1}\frac{M_{I2}}{M}-\mu_HH_{E12}-\sigma_{H2}H_{E12}
\end{equation}


\begin{equation}
    \label{eq1}
    \frac{dH_{E21}}{dt}={\lambda\beta}_{H1}H_{R2}\frac{M_{I1}}{M}-\mu_HH_{E21}-\sigma_{H1}H_{E21}
\end{equation}


\begin{equation}
    \label{eq1}
    \frac{dH_{I12}}{dt}=\sigma_{H2}H_{E12}-\mu_HH_{I12}-\alpha_2H_{I12}
\end{equation}


\begin{equation}
    \label{eq1}
    \frac{dH_{I21}}{dt}=\sigma_{H1}H_{E21}-\mu_HH_{I21}-\alpha_1H_{I21}
\end{equation}


\begin{equation}
    \label{eq1}
    \frac{dH_R}{dt}=\alpha_2H_{I12}+\alpha_1H_{I21}-\mu_HH_R+vH_S
\end{equation}




%%%%%%%%%%%%%%%%

%\subsection{Tablas de datos}
\begin{figure}[h]
\includegraphics[scale=0.9]{articulo/modelo_nulo_1.png}
\centering 
\caption{Condiciones iniciales y valores de parámetros del modelo nulo.}
\label{Fi:valores modelo nulo1}
\end{figure}


\begin{figure}[h]
\includegraphics[scale=0.9]{articulo/modelo_nulo_2.png}
\centering 
\caption{Condiciones iniciales y valores de parámetros del modelo nulo.}
\label{Fi:valores modelo nulo2}
\end{figure}


\begin{figure}[h]
\includegraphics[scale=0.9]{articulo/modelo_nulo_3.png}
\centering 
\caption{Condiciones iniciales y valores de parámetros del modelo nulo.}
\label{Fi:valores modelo nulo3}
\end{figure}

\begin{figure}[h]
\includegraphics[scale=0.9]{articulo/modelo_nulo_4.png}
\centering 
\caption{Condiciones iniciales y valores de parámetros del modelo nulo.}
\label{Fi:valores modelo nulo4}
\end{figure}

\begin{figure}[h]
\includegraphics[scale=0.9]{articulo/valores_controles.png}
\centering 
\caption{Parámetros de los controles.}
\label{Fi:val_controles}
\end{figure}


%\subsection{Diagramas de bloques}

\begin{figure}[h]
\includegraphics[scale=0.6]{articulo/Modelo con controles mosquitos.png}
\centering 
\caption{Diagrama de bloques del sistema con los controles para la población de mosquitos}
\label{Fi:diag_controles1}
\end{figure}

\begin{figure}[h]
\includegraphics[scale=0.6]{articulo/Modelo con controles humanos.png}
\centering 
\caption{Diagrama de bloques del sistema con los controles para la población de humanos}
\label{Fi:diag_controles2}
\end{figure}


%%%%%%%%%%%%%%%%%%%%%%%%%%%%%%%%%%%%%%%%%%%%%%%%%%%%%%%%%%%%%%%%%%%%%%

\begin{figure}[H]
\centering
\includegraphics[width=0.40\textwidth]{articulo/Simulación modelo nulo.png}
\caption{Modelo SEIR nulo: transmisión de dengue en humanos para el periodo Diciembre-Septiembre para cada variable.}
\label{fig:resultado1}
\centering
\end{figure}



\begin{figure}[h]
\includegraphics[scale=0.6]{articulo/Dengue Humanos Mecanico.png}
\centering 
\caption{Modelo computacional para el comportamiento de la enfermedad del dengue en la población humana con un control mecánico}
\label{3J}
\end{figure}
\begin{figure}[h]
\includegraphics[scale=0.6]{articulo/Dengue Mosquitos Mecanico.png}
\centering 
\caption{Modelo computacional para el comportamiento de la enfermedad del dengue en la población de mosquitos con un control mecánico}
\label{4J}
\end{figure}
\begin{figure}[h]
\includegraphics[scale=0.6]{articulo/Dengue Humanos Mecanico + Insecticida.png}
\centering 
\caption{Modelo computacional para el comportamiento de la enfermedad del dengue en la población humana con un control mecánico y uso de insecticida}
\label{5J}
\end{figure}
\begin{figure}[h]
\includegraphics[scale=0.6]{articulo/Dengue mosquitos Mecanico+Insecticida.png}
\centering 
\caption{Modelo computacional para el comportamiento de la enfermedad del dengue en la población de mosquitos con un control mecánico y uso de insecticida}
\label{6J}
\end{figure}
\begin{figure}[h]
\includegraphics[scale=0.6]{articulo/Dengue Humanos Mecanico + Larvicida.png}
\centering 
\caption{Modelo computacional para el comportamiento de la enfermedad del dengue en la población humana con un control mecánico y uso de Larvicida}
\label{7J}
\end{figure}
\begin{figure}[h]
\includegraphics[scale=0.6]{articulo/Dengue Mosquitos Mecanico+Larvicida.png}
\centering 
\caption{Modelo computacional para el comportamiento de la enfermedad del dengue en la población de mosquitos con un control mecánico y uso de larvicida}
\label{8J}
\end{figure}
\begin{figure}[h]
\includegraphics[scale=0.6]{articulo/Dengue Humanos Mecanico + Vacunacion.png}
\centering 
\caption{Modelo computacional para el comportamiento de la enfermedad del dengue en la población humana con un control mecánico y vacunación}
\label{9J}
\end{figure}
\begin{figure}[h]
\includegraphics[scale=0.6]{articulo/Dengue mosquitos Mecanico + Vacunacion.png}
\centering 
\caption{Modelo computacional para el comportamiento de la enfermedad del dengue en la población de mosquitos con un control de uso de control mecanico y vacunación.}
\label{11J}
\end{figure}
\begin{figure}[h]
\includegraphics[scale=0.6]{articulo/Dengue Humanos Insecticida+Larvicida.png}
\centering 
\caption{Modelo computacional para el comportamiento de la enfermedad del dengue en la población humana con un control de uso de insecticida y larvicida}
\label{12J}
\end{figure}
\begin{figure}[h]
\includegraphics[scale=0.6]{articulo/Dengue Mosquitos Inceticida+Larvicida.png}
\centering 
\caption{Modelo computacional para el comportamiento de la enfermedad del dengue en la población de mosquitos con un control de uso de Insecticida y uso de larvicida}
\label{10J}
\end{figure}
\begin{figure}[h]
\includegraphics[scale=0.6]{articulo/Dengue Humanos Insecticida + Vacunacion.png}
\centering 
\caption{Modelo computacional para el comportamiento de la enfermedad del dengue en la población humana con un control por uso de insecticida y vacunación}
\label{13J}
\end{figure}
\begin{figure}[h]
\includegraphics[scale=0.6]{articulo/Dengue mosquitos Incecticida + Vacunacion.png}
\centering 
\caption{Modelo computacional para el comportamiento de la enfermedad del dengue en la población de mosquitos con un control de uso de Insecticida y uso de vacunacion}
\label{14J}
\end{figure}
\begin{figure}[h]
\includegraphics[scale=0.6]{articulo/Dengue Humanos Larvicida.png}
\centering 
\caption{Modelo computacional para el comportamiento de la enfermedad del dengue en la población humana con un control por uso de Larvicida}
\label{15J}
\end{figure}
\begin{figure}[h]
\includegraphics[scale=0.6]{articulo/Dengue Mosquitos Larvicida.png}
\centering 
\caption{Modelo computacional para el comportamiento de la enfermedad del dengue en la población de mosquitos con un control de uso de Larvicida}
\label{16J}
\end{figure}
\begin{figure}[h]
\includegraphics[scale=0.6]{articulo/Dengue Humanos Larvicida + Vacunacion.png}
\centering 
\caption{Modelo computacional para el comportamiento de la enfermedad del dengue en la población humana con un control por uso de Larvicida y vacunación}
\label{17J}
\end{figure}
\begin{figure}[h]
\includegraphics[scale=0.6]{articulo/Dengue Mosquitos Larvicida + Vacunacion.png}
\centering 
\caption{Modelo computacional para el comportamiento de la enfermedad del dengue en la población de mosquitos con un control de uso de Larvicida y vacunación}
\label{18J}
\end{figure}
\begin{figure}[h]
\includegraphics[scale=0.6]{articulo/Dengue Humanos Vacunacion.png}
\centering 
\caption{Modelo computacional para el comportamiento de la enfermedad del dengue en la población humana con un control por uso vacunación}
\label{19J}
\end{figure}
\begin{figure}[h]
\includegraphics[scale=0.6]{articulo/Dengue mosquitos vacunacion.png}
\centering 
\caption{Modelo computacional para el comportamiento de la enfermedad del dengue en la población de mosquitos con un control de uso de vacunación}
\label{20J}
\end{figure}
\begin{figure}[h]
\includegraphics[scale=0.6]{articulo/medan.jpg}
\centering 
\caption{Resultado de simulaciones obtenidas en el estudio disponible en \cite{medan}, el cual ilustra la dinámica poblacional del dengue en Medan}
\label{1Jo}
\end{figure}

\end{document}